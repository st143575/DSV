\documentclass[10pt,a4paper]{article}
\usepackage[utf8]{inputenc}
\usepackage{longtable}
\usepackage{titlesec}
\usepackage{amsfonts}
\usepackage{graphicx}
\usepackage{hyperref}
\usepackage{amsmath}
\usepackage{amssymb}
\usepackage{multicol}
\usepackage{multirow}
\usepackage{array}
\usepackage{color}
\usepackage[dvipsnames]{xcolor}
\usepackage{float}
\usepackage[left=2cm, right=2.60cm, top=2cm, bottom=2.50cm]{geometry}
\usepackage[txtcentered=true, height=40pt, width=100pt]{thumbs}
\usepackage[german]{babel}
\usepackage[linewidth=1pt]{mdframed}
\usepackage{subfigure}
\usepackage[justification=centering]{caption}

\newcommand{\fancythumb}[2]{
	\addthumb{#1}{\large\sffamily\textbf{\space\space#1\vspace{5pt}}}{white}{#2}
}

\newcommand{\fancyformula}[2]{
	\small
	\raggedright\sffamily\textbf{#1}
	#2
}

\newcommand{\ftransform}{
	~\xrightarrow{~\mathcal{F}~}~
}

\DeclareMathOperator{\sinc}{sinc}
\DeclareMathOperator{\sgn}{sgn}
\DeclareMathOperator{\rect}{rect}
\DeclareMathOperator{\tri}{tri}
\renewcommand{\Re}{\operatorname{Re}}
\renewcommand{\Im}{\operatorname{Im}}

\pagenumbering{arabic}
\titleformat*{\section}{\sffamily\Large\bfseries}
\titleformat*{\subsection}{\sffamily\large\bfseries}
\titleformat*{\subsubsection}{\sffamily\normalsize\bfseries}





\begin{document}
\section*{Grundlagen}
\fancythumb{Grundlagen}{red}
\subsection*{Zeitdiskrete Signale}
% ......
\subsection*{Zeitdiskrete Systeme}
% ......
\subsection*{Impulsantwort und Faltung}
% ......
\subsection*{FIR und IIR}
% ......
\subsection*{Fourierreihe und Fouriertransformation}
% ......


\newpage
\section*{z-Transformation}
\fancythumb{z-Transform}{orange}


\newpage
\section*{Digitale Filter}
\fancythumb{Digitale Filter}{yellow}


\newpage
\section*{Korrelation}
\fancythumb{Korrelation}{green}


\newpage
\section*{DFT und FFT}
\fancythumb{DFT und FFT}{cyan}


\newpage
\section*{Spektralanalyse}
\fancythumb{Spektralanalyse}{blue}


\newpage
\section*{Mathe}
\fancythumb{Mathe}{violet}


\end{document}